\section{Introduction}
\label{sec:intro}

Considering the formalism presented in \cite{martin_ian_groupoids2006,wang2021}, 
given a coupled cell network containing $n$ nodes where the dynamical state of each node 
is represented by a variable $x_i$ and its internal dynamics is given by the
nonlinear function $f_i(x_1, \dots, x_n)$, we have that the dynamics of the 
system is represented by 
\begin{equation} \label{eq:system}
    \dot{\vec{x}} = \vec{F}(\vec{x}),
\end{equation}
where $\vec{x} = (x_1, \dots, x_n)$ and $\vec{F} = (f_1, \dots, f_n)$. To consider
the phenomenon of homeostasis, we add to the system of equations an input parameter
$I \in \mathbb{R}$ and also assume that the system has a stable equilibrium at 
$\vec{x} = \vec{x_0}$ and $I = I_0$ such that we have 
\begin{equation} \label{eq:system-input}
    \dot{\vec{x}} = \vec{F}(\vec{x}(I), I),
\end{equation}
and
\begin{equation}
    \vec{F}(\vec{x_0}(I_0), I_0) = 0.
\end{equation}

Following \cite{wang2021,homeostasis_antonelli2018}, we verify the possibility of existence of infinitesimal
homeostasis points and, further, the existence of infinitesimal chair points. For this, we
first choose one variable $z(I) = x_j(I)$ of the node variables $\vec{x}$ and define this 
variable as the \textbf{input-output map}\cite{homeostasis_antonelli2018}. The goal is to 
verify the homeostasis behavior in $z(I)$ as we vary the parameter $I$ around $I_0$, and 
this behavior is possible when we satisfy the following conditions:

\textbf{infinitesimal homeostasis}:
\begin{equation} \label{eq:infinitesimal-condition}
    \frac{\partial z}{\partial I}(I_0) = 0
\end{equation}

\textbf{infinitesimal chair}:
\begin{equation} \label{eq:chair-condition}
    \begin{aligned}
    \frac{\partial z}{\partial I}(I_0) = \frac{\partial^2 z}{\partial I^2}(I_0)& = 0, \\
    \frac{\partial^3 z}{\partial I^3}(I_0) &\neq 0.
    \end{aligned}
\end{equation}

Fortunately, the framework presented in \cite{wang2021} provides a consistent 
method to obtain the defining conditions for both infinitesimal homeostasis and 
infinitesimal chairs in a given coupled cell system. Although the authors of 
\cite{multi_input_antoneli2020} provides a generalization for multiple inputs, 
here we first consider that the input $I$ is provided to only one node of the network. 
The node depending directly on the input $I$ is called \textbf{input node}, while 
the node $j$ on which we define the input-output map $z(I)$ is denoted as 
\textbf{output node}. The rest of the nodes in the network are the 
\textbf{regulatory nodes}.

Denoting the input and output nodes as $\mathcal{I}$ and $\mathcal{O}$, respectively, 
while all the regulatory nodes as $\rho$, we consider the general Jacobian of the 
system of Eq.~\ref{eq:system} as 
\begin{equation}
    J = 
    \begin{pmatrix}
        f_{\mathcal{I}, \mathcal{I}} & f_{\mathcal{I}, \rho} & f_{\mathcal{I}, \mathcal{O}} \\
        f_{\rho, \mathcal{I}} & f_{\rho, \rho} & f_{\rho, \mathcal{O}}\\
        f_{\mathcal{O}, \mathcal{I}} & f_{\mathcal{O}, \rho} & f_{\mathcal{O}, \mathcal{O}}
    \end{pmatrix},
\end{equation}
where $f_{i,j} = \nicefrac{\partial f_i}{\partial x_j}$. Then, we define the homeostasis matrix 
$H$ as the Jacobian at the equilibrium point $(\vec{x_0}(I_0), I_0)$ without the first row and 
the last column (regarding the input and output nodes) \cite{wang2021}. Therefore,
\begin{equation}
    H = 
    \begin{pmatrix}
        f_{\rho, \mathcal{I}} & f_{\rho, \rho} \\
        f_{\mathcal{O}, \mathcal{I}} & f_{\mathcal{O}, \rho} 
    \end{pmatrix}.
\end{equation}

It follows from \cite{wang2021} that the conditions of infinitesimal homeostasis and
infinitesimal chair displayed by equations \ref{eq:infinitesimal-condition} and 
\ref{eq:chair-condition} can be obtained by using the matrix $H$. Thus, denoting
\begin{equation}
    det(H)(\vec{x_0}, I_0) = h(I_0),
\end{equation}
we have that

\textbf{infinitesimal homeostasis}:
\begin{equation} \label{eq:mat-infinitesimal-condition}
    h(I_0) = 0
\end{equation}

\textbf{infinitesimal chair}:
\begin{equation} \label{eq:mat-chair-condition}
    \begin{aligned}
        h(I_0) = h'(I_0) = 0 \\
        h''(I_0) \neq 0
    \end{aligned}
\end{equation}

For general circuits it is possible to obtain the defining conditions for 
both infinitesimal homeostasis and infinitesimal chair by using the topology
of the network and avoiding to solve the determinant of a $(n-1)\times(n-1)$
matrix. This idea is showed by the following equation:
\begin{equation}
    det(H) = det(B_1)det(B_2)\cdots det(B_m),
\end{equation}
where each $B_j$ is a $K \times K$ block submatrix of the matrix $H$ and its 
determinant is an irreducible polynomial.The connection with the network 
topology is provided by showing that for each $B_j$ there is a corresponding 
subnetwork of the original network \cite{wang2021}. Therefore, each 
$det(B_j) = 0$ defines a homeostasis condition.